\documentclass{sty/eiiatfg}

% Hay muchos paquetes para LaTeX que facilitan centenares de trabajos
% engorrosos. Busca en Internet si no sabes hacer algo con LaTeX. El
% repositorio oficial está en https://ctan.org/

% Algunos paquetes especialmente útiles ya están incluídos en el estilo
% del TFG (consulta sty/eiiatfg.cls para más detalles).

\input{datos-tfg.tex}
% Definición de acrónimos:
% \newacronym{id}{corto}{largo}
% o bien, en versión simplificada
% \acro{corto}{largo}

% Uso de acrónimos
% \acrshort{id} nombre corto
% \acrlong{id}  nombre largo
% \acrfull{id}  nombre largo (nombre corto)
% o bien, en versión simplificada
% \acs{id} nombre corto
% \acl{id}  nombre largo
% \ac{id}  nombre larco (nombre corto)

\acro{CD}     {Compact Disc}
\acro{GNU}    {\acs{GNU} is Not Unix}
\acro{PDF}    {Portable Document Format}
\acro{TCP/IP} {Pila de protocolos de Internet}
\acro{TCP}    {Transport Control Protocol}
\acro{XML}    {eXtensible Markup Language}


% La bibliografía la puedes descomponer en varios archivos .bib
% Los archivos .bib se pueden escribir a mano con ayuda de un editor online
% (e.g. http://truben.no/latex/bibtex) o generar con Mendeley u otro
% gestor de bibliografía. Solo se incluyen las referencias que son citadas
% en el texto.
\addbibresource{bib/main.bib}
\addbibresource{bib/how.bib}
\addbibresource{bib/ejemplos.bib}

\begin{document}

% Puedes cambiar la licencia de este documento con la orden license.  Por defecto se asume Creative Commons Attribution 4.0 (ver https://creativecommons.org/licenses/by/4.0/).  
% Por ejemplo, para restringir su uso, copia y distribución:
%\license{Todos los derechos reservados.}

% Si usas muchos símbolos conviene que describas lo que significan en un archivo
% aparte (en este caso simbolos.tex).  Si no es el caso puedes comentar esta línea
\listofsymbols{simbolos.tex}

\portada

% Edita los
\begin{agradecimientos}
\noindent Pon aquí tus agradecimientos, pero no te olvides de que se trata de un documento profesional.
\end{agradecimientos}	   
\begin{dedicatoria}
Aquí va la dedicatoria.\avisoLocalizacionArchivo
\end{dedicatoria}
\begin{resumen}

\avisoLocalizacionArchivo

\noindent Aquí va el resumen en español. Debería ocupar entre 1000 y 1500 palabras a modo orientativo.

\warning{Escribe el resumen en estilo periodístico, desde lo más general a lo más específico.  Desde lo más importante a lo menos importante.  Esto será especialmente útil si es necesario hacer un resumen más breve en otro contexto (presentación, herramienta, etc).  Si el resumen está en formato periodístico bastará con seleccionar los primeros párrafos.}
\end{resumen}

\begin{abstract}
\noindent  Write here the abstract in english. It must be between 1000 and 1500 words long.
\end{abstract}


\indices

% El cuerpo del documento está en la carpeta tex
% Aquí simplemente se incluyen los archivos correspondientes a cada capítulo.

% No los llames capitulo1, capitulo2, etc. Los números los pondrá LaTeX según
% el orden en que los pongas.  Eso facilitará después su posible reordenación
% o división.

% Esta estructura corresponde a un documento científico-técnico.
% Si ves que tu proyecto concuerda con un trabajo profesional organiza el 
% documento según UNE 157001

\chapter{Introducción} 
\label{ch:introduccion}

\avisoLocalizacionArchivo 

En este capítulo debes introducir el problema sin divagar, sin copiar de otros documentos y sin utilizar un lenguaje excesivamente técnico.  Tampoco utilices un lenguaje informal.  Este capítulo debería convencer al cliente de que el proyecto merece la pena.  Es decir, es un problema real y no está resuelto completamente.

Debe tenerse siempre presente que el cliente es el que paga.  En el TFG el que paga es el tribunal, en forma de calificación.  Así que a quien hay que convencer es a los miembros del tribunal.  El tribunal no lo conocerás a priori.  Por eso la memoria debe estar escrita para que la entienda alguien que no es especialista en el campo de aplicación.  Pero eso no implica que se toleren la falta de rigor o la falta de argumentación técnica.  Solo implica que los argumentos específicos hay que explicarlos o citar la fuente que los explica.

Redacta la introducción al final del TFG, cuando tengas elaborado el capítulo de antecedentes, los resultados y su discusión.  De esta forma podrás evitar repetir argumentos que ya están en esos capítulos.  La introducción debe introducir también el contexto en el que se desarrolla el TFG.  Divide el documento en secciones y subsecciones para organizar el contenido del capítulo.  Utiliza preferentemente frases cortas.


\section{Organización de la memoria} 
\label{sec:organizacion-memoria}

La organización de este documento responde a un documento científico-técnico. Se descompone en los siguientes capítulos.

\begin{description}
    \item[\autoref{ch:objetivos}] Enumera y justifica los objetivos del proyecto y establece los límites intrínsecos y extrínsecos de ejecución del TFG.
    \item[\autoref{ch:antecedentes}] Analiza los antecedentes y estado del arte en relación al tema del proyecto.
    \item[\autoref{ch:desarrollo}] Describe todo el proceso de desarrollo del TFG.  Esto incluye la metodología de trabajo empleada y las diferentes etapas o iteraciones que se han llevado a cabo.  No dudes en descomponer el capítulo en varios si aglutina demasiado material.
    \item[\autoref{ch:resultados}] Describe en detalle los resultados obtenidos y las pruebas realizadas. Discute los resultados en relación a los objetivos del proyecto.
    \item[\autoref{ch:conclusiones}] Recopila las principales conclusiones del proyecto y comenta las líneas de trabajo futuro, en caso de que se contemplen.
    \item[\deschyperlink{ch:anexos}{Anexos}] Complementan la información del cuerpo del documento con información técnica útil para reproducir los resultados, pero innecesaria para comprender en su totalidad el TFG realizado.
    \item[\deschyperlink{ch:bibliografia}{Bibliografía}] Recopila las referencias bibliográficas utilizadas en este documento.
\end{description}

\warning{Al finalizar el resto de los capítulos revisa esta descripción del documento para que coincida con lo que realmente contiene la memoria.  Por ejemplo, es frecuente fusionar varios capítulos en uno cuando son muy pequeños.  También es frecuente lo contrario, dividir un capítulo en varios cuando es muy extenso.}

\section{Repositorio de información}
\label{sec:repositorio}

\warning{Es muy útil tanto para la ejecución como para la evaluación disponer de un repositorio para almacenar las sucesivas versiones del documento y de todo el material generado durante el proyecto.  La UCLM está gestionando la incorporación de \href{https://github.com}{GitHub} como servicio institucional.  Otras posibilidades son \href{https://gitlab.com}{GitLab} o \href{https://bitbucket.org}{BitBucket}. Si no utilizas un repositorio quita esta sección.}

Todo el material generado durante la ejecución de este proyecto está disponible en el repositorio \thegitrepo{}.  El material incluye el código \LaTeX{} del presente documento, el código fuente de los programas realizados o modificados, y todos los datos generados en la evaluación de resultados.
\input{objetivos.tex}
\input{antecedentes.tex}
\chapter{Desarrollo del TFG}
\label{ch:desarrollo}

\avisoLocalizacionArchivo 

En general este capítulo debe describir cómo se ha llegado a la solución del problema incluyendo metodología, planificación y ejecución.

En primer lugar este capítulo debe tratar con la metodología de trabajo empleada para elaborar el TFG.  La metodología puede variar sustancialmente dependiendo del contexto o el área temática en la que se encuadre.  Si es necesario, divide la descripción de la metodología como capítulo independiente.

A continuación debe describir cómo se ha dividido el problema en una secuencia de fases, tareas o iteraciones y qué cantidad de recursos (tiempo fundamentalmente) se han planificado para cada una.

Por último, debe incluir información sobre la ejecución del proyecto. Cuáles han sido los problemas encontrados y qué desviaciones se han producido en la ejecución.  Dependiendo de la naturaleza del proyecto es posible organizar esta información de manera conjunta con la planificación.

\chapter{Resultados}
\label{ch:resultados}

\avisoLocalizacionArchivo 

Escribe en este capítulo los resultados del proyecto.  Este capítulo debería explicar los resultados de forma global, no los resultados de cada fase o iteración.  Probablemente será el capítulo con más tablas y gráficas.  Revisa las secciones~\ref{sec:figuras} y~\ref{sec:tablas} para aprender cómo se escriben en \LaTeX{}.

Tus contribuciones no tienen por qué limitarse al trabajo sistemático del TFG.  Puede que hayas contribuido en aspectos metodológicos, en ideas novedosas, en la planificación de experimentos, en desarrollos matemáticos. Este capítulo está para agrupar todo eso.  Describe con claridad todo lo que ha supuesto contribuciones originales por tu parte.

\warning{Es importante destacar que un resultado negativo es también un resultado.  Es posible que el proyecto planteara abordar un problema con un método que ha demostrado ser no apto.  Si el trabajo ha sido sistemático sigue teniendo mucho valor, puesto que excluye el método para cualquier otro trabajo futuro.  Escribe este tipo de resultados con especial cuidado para destacar que el trabajo se ha realizado de manera sistemática.}
\chapter{Conclusiones}
\label{ch:conclusiones}

\avisoLocalizacionArchivo 

Las conclusiones deben cerrar el documento, destacando los aspectos más importantes de la ejecución del TFG.  Debe analizar qué objetivos se han alcanzado y en qué grado, qué objetivos se han tenido que dejar fuera del proyecto y por qué, y qué líneas de trabajo futuro abre el TFG. 

Fíjate en que los objetivos abren el trabajo personal y las conclusiones lo cierran.  Procura mantener un orden que resalte esta relación, pero no te limites a parafrasear los objetivos.

% Los anexos pueden ir en la misma carpeta que el cuerpo del documento 
% porque eso facilita la migración de partes de un capítulo a un anexo.
\appendix\cleardoublepage
\hypertarget{ch:anexos}{%
    \input{latex.tex}}

% La bibliografía no suele ir numerada porque se pone después de los anexos.
% No se debe poner antes de los anexos porque si se cita una referencia en
% un anexo sería una backward reference, que deben evitarse a toda costa
\cleardoublepage
\hypertarget{ch:bibliografia}{%
    \printbibliography[heading=bibintoc,title={Bibliografía}]}
\cleardoublepage

\end{document}
